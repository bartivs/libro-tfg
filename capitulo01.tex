%^--- El primer capítulo es fijo y siempre se llama introducción
\chapter[Introducción]{Introducción}
Durante los últimos años, la Inteligencia Artificial (IA) ha tenido un crecimiento exponencial,
esto se debe principalmente a la capacidad de cómputo asequible y de alto rendimiento, además de
los grandes volúmenes de datos que se encuentran disponibles.\cite{Oracle}\\
\indent La IA tiene varias áreas de estudio, entre ellas se encuentra el Procesamiento de Lenguaje
Natural (NLP, por sus siglas en inglés), encargado de hacer que las computadoras entiendan e
interpreten el lenguaje humano con la ayuda de modelos estadísticos, machine learning y deep
learning.\\
\indent Una de las tantas apĺicaciones prácticas del NLP son los chatbots, programas que imitan la
conversación humana.Los chatbots son capaces de interactuar con personas y de responder
adecuadamente a sus preguntas, son bastante accesibles, eficientes y de alta disponibilidad,
permitiendo así que distintas industrias se beneficien de el, entre las que destacan el comercio
electrónico, los seguros y el cuidado de la salud .\cite{building_chat-bots-with-python}\\
\indent En este trabajo se presenta un chatbot para el sector de la educación, donde el estudiante
pueda realizar sus consultas académicas y la respuesta deberá ser generada de acuerdo a ciertas
técnicas de coincidencia de patrones.

\section{Objetivo General}
Desarrollar e implementar un Chatbot utilizando algoritmos de Inteligencia Artificial (IA).

\section{Objetivos Específicos}
\begin{itemize}
	\item Comparar distintas Tecnologías para abordar el problema.
	\item Orientar el chatbot a dudas comunes de los estudiantes de la Facultad de Ingeniería en
	      atención al alumno.
	\item Recopilar, procesar y filtrar preguntas frecuentes para contar con un dataset propio.
	\item Seleccionar, entrenar y probar el algoritmo utilizando el dataset generado.
	\item Implementar el chatbot para su uso por estudiantes de la Facultad de Ingeniería
	\item	Agregar una interfaz, propia o a una ya existente para el uso por los alumnos.
\end{itemize}

\section{Alcance y limitaciones}
Se pretende desarrollar un Chatbot que utilice inteligencia artificial para responder a preguntas
frecuentes de los alumnos de la Facultad de Ingeniería de la UNA. El entrenamiento de la IA se
llevara a cabo utilizando un dataset propio en conjunto con otros ya existentes.
Una previa comparativa entre distintas librerías y plataformas es necesaria para escoger la mejor
solución al objetivo planteado.
