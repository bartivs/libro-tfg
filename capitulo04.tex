\chapter[IMPLEMENTACIÓN]{IMPLEMENTACIÓN}
En los capítulos anteriores presentamos implementaciones de alto nivel de la arquitectura de
sistemas conceptuales de chatbots, así como también arquitecturas recomendadas por la documentación
de Rasa. En el presente capítulo presentaremos los componentes elegidos y sus funciones en el
sistema.

\section{Docker}
Existen varias implementaciones de contenedores para servidores como Docker, containerd, OpenVZ y
HyperV. La más utilizada en la industria y con más variedad de servicios actualmente es
Docker por lo que decidimos elegir esta herramienta para el desarrollo y despliegue de servicios del
proyecto. Docker puede ser ejecutado sobre sistemas basados en Windows, Linux y macOS lo que
facilita la igualdad de condiciones entre ambientes de desarrollo y producción, ademas es de uso
libre y gratuito. \cite{alternativas_docker}

\subsection{Contenedor}

Los contenedores son la unidad más pequeña un sistema, es una entidad que se utiliza para aislar
cada componente del sistema base. Cada contenedor puede aislarse mediante funciones del sistema
operativo llamadas cgroups y cnames logrando así aplicaciones en entornos aislados (sandbox en
Ingles). \cite{Docker}

\begin{figure}[ht]
	\centering
	\includegraphics[width=\textwidth]{imagenes/cap4/docker-container.png}
	\caption{Contenedores}
	\cite{Docker}
	\label{fig:container_diagram}
\end{figure}

\subsection{Imagen de contenedor}

Una imagen de contenedor es el sistema de archivos aislados de todos los archivos necesarios para
ejecutar la aplicación, así como dependencias, configuraciones, ejecutables, etc. Así como también
variables de entorno y datos necesarios para ejecutar la aplicación. \cite{Docker}

\subsection{Redes}

Entre las ventajas de desplegar una aplicación por medio de contenedores Docker es que se pueden
comunicar entre ellos y también con servicios externos al entorno de Docker. Por defecto se pueden
crear varios tipos de configuraciones de red, pero por lo general se utilizan redes puente entre
los contenedores para que estos puedan comunicarse mutuamente entre ellos y solo exponiendo los
puertos necesarios para interactuar con el sistema en cuestión y esta es la opción utilizada en
nuestra implementación. \cite{Docker}

\subsection{Volúmenes}
Puesto que un contenedor no tiene un estado persistente sobre los datos que genera, se introducen
los volúmenes, son la forma recomendada de agregar persistencia de datos a un contendedor de
Docker.
\cite{Docker}

\begin{figure}[ht]
	\centering
	\includegraphics[width=\textwidth]{imagenes/cap4/docker-volume.png}
	\caption{Volumen}
	\cite{Docker}
	\label{fig:volume_diagram}
\end{figure}

\subsection{Construcción}
Las imágenes de Docker se construyen partir de instrucciones escritas en un archivo denominado
Dockerfile,
generalmente de parte de una imagen base de la cual sé la adiciona lo necesario para ejecutar la
aplicación.
\cite{Docker}

\subsection{Repositorios}
Un repositorio de imágenes Docker(Docker Registry) es un servidor que almacena y distribuye
imágenes versionadas generadas partir de un Dockerfile. Estos repositorios pueden ser públicos como
DockerHub que es el oficial de la comunidad de Docker como así también privado para un equipo de
desarrollo en una institución.
\cite{Docker}

\section{Componentes}

Cada componente del sistema se configuró en un contenedor de Docker con la excepción del servidor
NGINX que si estaba instalado sobre el sistema operativo del servidor.
\begin{figure}[ht]
	\centering
	\includegraphics[width=\textwidth]{imagenes/cap4/server.png}
	\caption{Componentes del sistema}
	\textbf{Fuente:} Elaboración propia.
	\label{fig:server_diagram}
\end{figure}

\subsection{Rasa Open Source}

El contendedor de Rasa Open Source ejecuta una imagen oficial proveída por Rasa disponible en los
repositorios de DockerHub\cite{DockerHub}, pero con el modelo entrenado y las configuraciones
particulares al proyecto. Es el único contenedor que tiene un puerto externo para servir al
usuario. También hace uso de los servicios de PosgresSQL y del servicio de acciones(action server).

\subsection{Servidor de acciones}

Servidor de acciones(acction server) es un servicio interno que ejecuta código escrito en Python
este utiliza una imagen oficial para servicios Python disponible en los repositorios de
DockerHub\cite{DockerHub}, estos son operaciones específicas para algunas acciones una respuesta no
estática, como por el ejemplo la acción relacionada con el cálculo de puntajes en el final
desacuerdo a los puntajes en los exámenes parciales.

\subsection{PostgresSQL}
En la documentación de Rasa, especifica que se necesita un almacenamiento para el registro de
eventos (event broker) para su posterior análisis o también para el procesamiento de los eventos
por otros
servicios. Dependiendo de la demanda y el flujo de datos se pueden elegir Pika, Kafka y asi como
el uso de Bases de datos basadas en SQL para el almacenamiento. \cite{event_broker} Como
contábamos con experiencia previa en el uso de bases de datos utilizamos PostgresSQL que era una
implementación ya conocida, ademas para el desarrollo las funcionalidades adicionales de  si bien
podrían haber sido útiles agregarían complejidad innecesaria al sistema en fases iniciales y pueden
ser adicionados posteriormente.
PostgresSQL es un motor de base de datos del tipo relacional\cite{postgresql} el cual se configuró
a partir de una imagen oficial de PostgresSQL disponible en los repositorios de
DockerHub\cite{DockerHub}. Aparte de sus funciones de base de datos de sistema, también ejecutar un
trabajo periódico(cada una hora) para realizar una copia actualizada de los contenidos de la tabla
Eventos a un archivo separado por comas(csv) que se utiliza para retroalimentar las conversaciones
y generar más datos para mejorar el modelo.

\subsection{NGINX}
Para poder servir los archivos y la redirección interna de  las peticiones al servidor de Rasa Open
Source se necesita un servidor WEB con estas características. Existen  varias implementaciones que
pueden ser útiles como Apache, Caddy y NGINX. El más utilizado en la industria es NGINX por lo que
elegimos por su facilidad para configuración, amplia disponibilidad de documentación y asi como
también ser el más eficiente entre las demás opciones. \cite{web_servers}
NGINX, es un servidor web que también puede ser usado como proxy reverso, que implica redirigir el
tráfico a puertos internos y también para servir archivos que fueron las funciones utilizadas para
el proyecto. Así como también puede ser utilizado como balanceador de carga, mail proxy y HTTP
cache, entre otras funciones \cite{NGINX}
El servidor NGINX redirige el tráfico a la instancia de Rasa Open Source y si como también sirve el
archivo de
la copia más reciente de la tabla Eventos para agilizar las verificaciones de las respuestas y
preguntas recibidas.

\section{Recursos}
Para la puesta en producción se necesita un servidor que este disponible con acceso a internet y a
una IP publica, dada la amplia variedad de opciones de proveedores de nubes publicas, las ventajas
en disponibilidad, accesibilidad, facilidad cambio de prestaciones y el bajo
coste de mantenimiento que tienen actualmente en comparación de implementar un servidor propio
local optamos por usar un servicio en la nube. \cite{cloud_providers}

Se eligió el proveedor DigitalOcean, se necesito una instancia de Droplet alojado en Nueva
York del proveedor DigitalOcean con un procesador con un 1vcpu(procesador virtualizado), 1 GB de
RAM, 25 GB de almacenamiento y con un costo de entre 4 y 6 dólares americanos al mes dependiendo
del tráfico presentado. Por las limitaciones de memoria del servidor se configuró también 4 GB de
espacio de intercambio(SWAP) en el espacio de almacenamiento.

\section{Telegram}
Entre las opciones de interfaz existen tres opciones populares, implementar un cliente web o
aplicación Movil, o integrarlo dentro de plataformas de mensajería existentes. Para evitar agregar
complejidad al sistema decidimos optar por el uso de una plataforma existente. Si bien la plataforma
de mensajería más utilizada en nuestro medio es WhatsApp, esta presenta
costos para el uso de su interfaz de programación por lo que fue descartada para la implementación del
proyecto, la segunda más utilizada es Telegram que debido a su una amplia comunidad
para desarrollo de soluciones dentro de la plataforma puesto que es bastante sencillo
usar su servicio para conectar a implementaciones de chatbots, ademas no requiere costos para el
uso de su interfaz de programación por lo que consideramos que era la mejor opción para
publicar la solución a los usuarios finales.
\cite{botfather}