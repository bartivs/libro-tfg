\chapter[CONCLUSIONES]{CONCLUSIONES}

Primeramente se hizo una revisión de las tecnologías y plataformas existentes para el desarrollo de chatbots,
mediante una comparativa finalmente se optó por Rasa Open Source 
 que ayuden a los alumnos de la Facultadde Ingeneiria
por ser de código libre, 
gran capacidad de personalización y además poseer buena documentación.
Para el entrenamiento inicial del modelo se utilizó un conjunto de datos propio, una vez que el modelo pudo ser desplegado en un servidor público
y el Bot haya estado disponible para los alumnos mediante una API de telegram, se recopilaron las preguntas frecuentes que tienen los estudiantes de la facultad de Ingeniería.
Las configuraciones del bot que contienen algoritmos de IA fueron seleccionadas de acuerdo a las necesidades del proyecto, estos fueron
entrenados y probados utilizando el conjunto de datos propio junto con el creado a partir de las preguntas de los alumnos.

Comparar distintas Tecnologías para abordar el problema.
Orientar el chatbot a dudas comunes de los estudiantes de la Facultad de Ingeniería en
atención al alumno.
Recopilar, procesar y filtrar preguntas frecuentes para contar con un dataset propio.
Seleccionar, entrenar y probar el algoritmo utilizando el dataset generado.
Implementar el chatbot para su uso por estudiantes de la Facultad de Ingeniería
Agregar una interfaz, propia o a una ya existente para el uso por los alumnos.