\chapter[ANALISIS]{ANALISIS DE RESULTADOS}

En este capítulo se presentan los resultados obtenidos en la implementación del chatbot utilizando
la plataforma Rasa OpenSource para responder preguntas frecuentes de estudiantes de la Facultad de
Ingeniería.  Además, se analizan los datos recopilados durante el entrenamiento del modelo, se
discuten las limitaciones, evaluara la efectividad del chatbot, asi como tambien posibles mejoras
en la implementación del chatbot.

\section{Configuraciones ulilizadas}
Dado el interés en que los resultados obtenidos sean replicables, en primer lugar, se explicarán
los componentes seleccionados para el entrenamiento y despliegue del modelo.
\begin{itemize}
	\item \textbf{WhitespaceTokenizer}: componente que divide el texto en palabras
	      individuales, utilizando espacios en blanco como delimitador.
	      \cite{Configuration_Documentation}
	\item \textbf{RegexFeaturizer}: componente que crea características basadas en expresiones
	      regulares. Esto puede ser útil para detectar patrones en el texto.
	      \cite{Configuration_Documentation}
	\item \textbf{LexicalSyntacticFeaturizer}: componente que combina características léxicas y
	      sintácticas para crear una mejor representación del texto. Utiliza etiquetas de
	      partes del discurso
	      y etiquetas de análisis de dependencia para crear características.
	      \cite{Configuration_Documentation}
	\item \textbf{CountVectorsFeaturizer}: componente que crea una representación dispersa de
	      bolsa de
	      palabras del texto. Puede ser utilizado para crear características para la
	      clasificación de
	      intenciones o la extracción de entidades. \cite{Configuration_Documentation}
	\item \textbf{DIETClassifier}: componente que combina una red neuronal recurrente con un
	      transformador para realizar la clasificación de intenciones y el reconocimiento de
	      entidades.
	      Utiliza múltiples fuentes de información, como incrustaciones de palabras,
	      incrustaciones de
	      caracteres y etiquetas de partes del discurso. \cite{Configuration_Documentation}
	\item \textbf{EntitySynonymMapper}: componente que mapea las entidades a su forma canónica.
	      Esto
	      puede ser útil para manejar variaciones en la forma en que se expresan las entidades
	      en el texto. \cite{Configuration_Documentation}
	\item \textbf{ResponseSelector}: componente que selecciona una respuesta basada en la
	      entrada del
	      usuario. Utiliza un enfoque basado en recuperación, donde coincide la entrada del
	      usuario con un
	      conjunto de respuestas predefinidas. Puede ser útil para manejar preguntas frecuentes
	      o
	      conversaciones informales.\cite{Configuration_Documentation}
	\item \textbf{FallbackClassifier}: componente que clasifica los mensajes como fallback si
	      no
	      coinciden con ninguna de las intenciones en el modelo. Puede ser útil para manejar
	      mensajes fuera
	      de contexto o solicitudes que el modelo no está entrenado para
	      manejar.\cite{Configuration_Documentation}
\end{itemize}

\section{Posibles Mejoras}

\begin{itemize}
	\item Integrar a sistemas existentes de la Facultad de Ingeniería.
	\item Autenticar usuarios a los sistemas de la Facultad de Ingeniería para obtener datos
	      especificos.
	\item Implementacion de interfaz propia o utilizacion de elementos interactivos como bottones
	      para respuestas rapidas.
	\item Se puede agregar respuestas de la chatbots publicos como  ChatGPT para preguntas fuera del
	      contexto de la Facultad de Ingeniería.
\end{itemize}
