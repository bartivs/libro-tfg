%%%%%%%%%%%%%%%%%%%%%%%%%%%%%%%%%%%%%%%%%%%%%%%%%%%%%%%%%
%    Formato de memoria de Trabajo Final de Grado (TFG) para la Facultad de Ingeniería de la Universidad Nacional de Asunción.

%   Version 2.0.1 por ProfMsc. Ing. Sergio Ramón Toledo Gallardo del 10 de junio de 2015                           
%- La compilación debe realizarse utilizando el comando pdflatex o pdflatexmk (para Mac)%
% Los programas necesarios son Miktex 2.9 (distribución latex) y algún procesador como el texmaks o texniccenter
%- Para texnicCenter Usar makeindex y habilitar usar bibtex
%-  Al imprimir el PDF se debe seleccionar tamaño real y Elegir origen por tamaño de pagina PDF para respetar los margenes requeridos   
%- Las figuras deben cargarse en la carpeta "imagenes" en formato pdf para optimizar la velocidad de compilación, sin embargo también se permiten formatos jpg y eps.
%- Las referencia bibliográficas se cargan en el archivo    fiuna.bib                                                       %%%%%%%%%%%%%%%%%%%%%%%%%%%%%%%%%%%%%%%%%%%%%%%%%%%%%
\documentclass[oneside, a4paper, 12pt]{fiuna}
\usepackage{fiuna}

%%%%%%%%%%%%%%%%%%%%%%%%%%%%%%%%%%%%%%%%%%%%%%%%%%%%%%%%%%
%               Inicio del documento                     %
%%%%%%%%%%%%%%%%%%%%%%%%%%%%%%%%%%%%%%%%%%%%%%%%%%%%%%%%%% 
\begin{document}

%%%Introduzca el nombre de los autores
\autor[Carlos Ozuna] {Carlos Buenaventura Ozuna Loncharich}
\autor[Oscar Valdez]{Oscar Bartolome Valdez Sarubbi}
%\autor{Autor 2}
%%%Introduzca el título del TFG
\titulo{Implementación de un ChatBot utilizando algoritmos de Inteligencia Artificial.}
%%%introduce la carrera
\carrera{Ingenier\'ia Electr\'onica}
%%%Nombre del Asesor del TFG %%%%%
\asesor{Prof. Ing. Aurora Nuñez}
%\asesor{Ing. Asesor 2}
%\asesor{Ing. Asesor 3}
%%%%%%Ciudad donde se presenta%%%%%%%%%%%
\ciudad{San Lorenzo}

%%%%%% A quien se dedique %%%%%%%%%%%%%%
\dedicatoria{A nuestros compañeros y profesores.}

\pagenumbering{roman}

\preliminares

\pagenumbering{arabic}
\tableofcontents
\listoffigures
\listoftables
%% En caso de no tener anexos comentar la siguiente linea
\listofappendices

\formatoTitulos

%%% Inician los capítulos de la memoria de TFG%%%%
%cada capitulo se introduce con \input
%El primer capitulo siempre es la introducción al TFG
%^--- El primer capítulo es fijo y siempre se llama introducción
\chapter[Introducción]{Introducción}
Durante los últimos años, la Inteligencia Artificial (IA) ha tenido un crecimiento exponencial, esto se debe principalmente a la capacidad de cómputo asequible y de alto rendimiento, además de los grandes volúmenes de datos que se encuentran disponibles.\cite{Oracle}\\
\indent La IA tiene varias áreas de estudio, entre ellas se encuentra el Procesamiento de Lenguaje Natural (NLP, por sus siglas en inglés), encargado de hacer que las computadoras entiendan e interpreten el lenguaje humano con la ayuda de modelos estadísticos, machine learning y deep learning.\\
 \indent Una de las tantas apĺicaciones prácticas del NLP son los chatbots, programas que imitan la conversación humana.Los chatbots son capaces de interactuar con personas y de responder adecuadamente a sus preguntas, son bastante accesibles, eficientes y de alta disponibilidad, permitiendo así que distintas industrias se beneficien de el, entre las que destacan el comercio electrónico, los seguros y el cuidado de la salud .\cite{building_chat-bots-with-python}\\
 \indent En este trabajo se presenta un chatbot para el sector de la educación, donde el estudiante pueda realizar sus consultas académicas y la respuesta deberá ser generada de acuerdo a ciertas técnicas de coincidencia de patrones.
 
\section{Objetivo General}
Desarrollar e implementar un Chatbot utilizando algoritmos de Inteligencia Artificial (IA).

\section{Objetivos Específicos}
\begin{itemize}
    \item Comparar distintas Tecnologías para abordar el problema.
    \item Orientar el chatbot a dudas comunes de los estudiantes de la Facultad de Ingeniería en atención al alumno.
    \item Recopilar, procesar y filtrar preguntas frecuentes para contar con un dataset propio.
    \item Seleccionar, entrenar y probar el algoritmo utilizando el dataset generado.
    \item Implementar el chatbot para su uso por estudiantes de la Facultad de Ingeniería
    \item	Agregar una interfaz, propia o a una ya existente para el uso por los alumnos.
\end{itemize}

\section{Alcance y limitaciones}
Se pretende desarrollar un Chatbot que utilice inteligencia artificial para responder a preguntas frecuentes de los alumnos de la Facultad de Ingeniería de la UNA. El entrenamiento de la IA se llevara a cabo utilizando un dataset propio en conjunto con otros ya existentes.
Una previa comparativa entre distintas librerías y plataformas es necesaria para escoger la mejor solución al objetivo planteado. 
%Luego inican los demas capítulos

\chapter[Revisión de literatura]{Marco Teórico}


\subsection[Reseña Histórica]{Reseña Histórica}
Para empezar a adentrarnos a los conceptos del Procesamiento de Lenguaje Natural primero empezaremos a repasar hitos importantes en áreas de conocimiento afines.
% https://www.cs.bham.ac.uk/~pjh/sem1a5/pt1/pt1_history.html
Los primera aplicación reconocida como una NPL fueron en 1948 que fue un buscador de palabras en el diccionario de desarrollado en Birkbeck College de Londres por Warren Weaver. Luego rápidamente surgió como idea de investigación crear una maquina de traducción automática en varios grupos de en Estados Unidos, Reino Unido, Francia y Rusia. Los primeros grupos se concentraron en traducir texto en Alemán, cuando los textos de la Segunda Guerra Mundial se volvieron obsoletos, pasaron a el Ruso el mayor problema fue que no existían todavía fundamentos formales de al computación, ni mucho menos de NLP. La mayoría de los investigadores eran matemáticos e inmigrantes bilingües que intentaban encontrar relaciones entre ambos lenguajes. La investigación concluyo en que todavía no existían los recursos necesarios para abordar dicho problema, pero se desarrollaron mejoras en herramientas para la traducción asistida.\cite{hancox}
%  Sumit raj
En 1950 Alan Turing desarrolla el test de Turing que es un test para distinguir el nivel de inteligencia de una maquina, propuso que un humano evaluara las conversaciones en un lenguaje natural entre un humano y una máquina diseñada para dar respuestas similares a los humanos. Lo que pudo definir la inteligencia de un sistema de computo de una forma comparable a la inteligencia humana.
En el año 1954 un experimentó de colaboración de la Georgetown University e IBM se  hace publica la primera demostración logra la traducción automática de mas de 60 se oraciones en Ruso al Ingles. Las oraciones eran previamente elegidas que trataban de temas políticos, legales, matemáticos y científicos. Luego eran ingresados a la maquina escritos en Ruso con letras romanizadas. El método era principalmente lexicográfico basado en un diccionario de relaciones de palabras de ambos idiomas.\cite{ibm_2003}
En las décadas posteriores se siguieron dando avances con métodos basados en reglas algorítmicas como arboles de decisión para el procesamiento del lenguaje y la traducción automática. A partir de los 2000s la investigación se centro en algoritmos de Machine Learning no supervisados y semi supervisados, por amplia disponibilidad de información no clasificados en literatura e internet, por lo general estos métodos son menos eficientes que los algoritmos supervisados pero la cantidad de datos existentes pueden equiparar estas deficiencias. En la segunda década del nuevo milenio se utilizaron nuevos métodos basados en Aprendizaje de características y Deep Learning lo que nos trae al estado actual del arte que discutiremos en la siguiente sección.




\section[Procesamiento de lenguaje natural]{Procesamiento de lenguaje natural (NPL)}
El procesamiento de lenguaje natural o por sus siglas en inglés NPL(de natural language processing) es un campo de las Ciencias de Computación y la Lingüística que trata con métodos para analizar, modelar y entender el lenguaje humano.


%---usar sownya vajja Practical Natural Language Processing_ A Comprehensive Guide to Building Real como referencia

\subsection{Lenguaje}% que es un lenguaje, fonemas, syntaxis, etc 
Primero de los conceptos claves para poder adentrarnos a  NLP es definir el Lenguaje, un Lenguaje es un sistema de comunicación que involucra una combinación compleja de componentes como, letras, palabras, etc. La Lingüística es el estudio sistemático del lenguaje. Para poder estudiar NLP, es importante entender componentes del lenguaje.\cite{sowmya_practical_npl} 
\begin{figure}[h]
    \centering
    \includegraphics[width=\textwidth]{imagenes/Cap 2/lenguaje.drawio.png}
    \caption{Aplicaciones de partes de un Lenguaje}
    \label{fig:Arquitectura}
    \cite{sowmya_practical_npl}
\end{figure}

%metodos heuristicos
\subsection{Métodos Heurísticos y basado en reglas}

Similarmente a otros métodos primitivos usando AI, los primeros intentos en el diseño de sistemas de NLP fueron
en construir reglas de acciones a manualmente por medio de arboles de decisión. Esto requiere que lo desarrolladores tengan
experiencia en el dominio 
del problema para formular las reglas que puedan ser incorporadas al programa.\\
En estos sistemas también se requieren recursos como diccionarios y tesauros, típicamente compilados y digitalizado. Otra 
herramienta muy poderosa en la implementación de sistemas basados en reglas son las Expresiones Regulares(Regex por sus
siglas en ingles). Una expresión regular es un conjunto de caracteres o patrón que es utilizada para coincidir y
encontrar subcadenas en un texto como  
\verb|^([a-zA-Z0-9_\-\.]+)@([a-zA-Z0-9_\-\.]+)\.([a-zA-Z]{2,5})$|
que es utilizado encontrar direcciones de email validas en un texto.\\
Las reglas y heuristica juegan un rol importande en todo el ciclo de vida de proyectos de NLP incluso hoy en dia. En un 
por un lado, estos son una buena manera de desarrollar primeras versiones. O también pueden ser de suma importancia en 
sistemas basados en AI para llenar vacíos o limitaciones de los modelos probabilísticos.
\cite{sowmya_practical_npl}

%Metodos basados en ML
\subsection{Métodos basados en Aprendizaje Automático (Machine Learning)}
El Aprendizaje Automático o ML(por sus siglas en ingles) son aplicados para datos de texto así como también son utilizados
en otro tipos de datos, como imágenes, voz y datos estructurados. Métodos de aprendizaje supervisados como clasificación y 
regresión con usados con mucha frecuencia en NLP. Como por ejemplo para buenas aplicaciones son par clasificar temas de 
artículos en el caso de calificadores.\\
Por otro lado las técnicas de regresión suelen ser utilizadas para dar una predicción numérica, como por ejemplo el precio
de un stock basado en discusiones en una red social.Similarmente métodos no supervisados pueden ser útiles para
agrupar documentos similares.
%deep learning
\subsection{Aprendizaje Profundo}

Aprendizaje profundo o Deep learning(DL) en ingles, es una evolución mas compleja de los algoritmos de ML convencionales,
se trata de combinaciones de nodos que emulan el funcionamiento de las redes neuronales del cerebro, sin entrar en mucho detalle procederemos a analizar los métodos basados en DL mas utilizados y exitosos actualmente en el área de NLP.

\subsection{Redes Neuronales Recurrentes(RNN)}

En todos los lenguajes una oración tiene una dirección de lectura, por ejemplo en el Castellano se lee de izquierda a derecha. Entonces un modelo que puede ser útil para leer progresivamente una entrada de texto podría ser muy útil para NLP. Las Redes Neuronales recurrentes o RNNs por sus siglas del ingles están específicamente diseñadas para mantener
un procesamiento secuencial y memoria de los pasos anteriores. Esta memoria es temporal y la información es almacenada y 
actualizada en cada paso de lectura de la RNN. 

\begin{figure}[h]
    \centering
    \includegraphics[width=\textwidth]{imagenes/Cap 2/rnn.png}
    \caption{Aplicaciones de partes de un Lenguaje}
    \label{fig:RNN}
    \cite{sowmya_practical_npl}
\end{figure}

Las RNNs son poderosa y funcionan muy bien para resolver muchas tareas de NLP, como clasificación de texto, reconocimiento
de entidad, tradición automática, etc. También se pueden utilizar para generar texto, por ejemplo para predecir la 
siguiente palabra de se va a escribir de acuerdo al contexto de lo que ya se escribió.

A pesar de su versatilidad y capacidad, las RNNs sufren de ciertas limitaciones por contar con memoria temporal, por lo 
tanto no se desempeñan óptimamente para textos con largos contextos. Pero para estos existen variaciones optimizadas para 
memoria de largo plazo(LSTMs). 



\subsection{Redes Neuronales Convencionales(CNN)}
Las CNNs por sus siglas en Ingles o Redes Neuronales Convoluciones son muy populares y muy frecuentemente usadas para 
para aplicaciones como clasificación de imágenes, reconocimiento de vídeo, etc. Las CNNs también vieron éxito en NLP, 
específicamente en clasificación de texto. Se puede reemplazar una palabra de una oración de un texto por un vector
palabra, y estos a su vez ser colocados en una matriz para poder ser tratados de manera similar a una imagen.

\begin{figure}[H]
    \centering
    \includegraphics[width=0.8\textwidth]{imagenes/Cap 2/cnn.png}
    \caption{Diagrama de CNN}
    \label{fig:RNN}
    \cite{sowmya_practical_npl}
\end{figure}

\subsection{Tranformadores}

Los trasformadores son la ultima innovación en lo que respecta a modelos basados en DL para NLP. Los 
modelos de transformadores obtuvieron avances en la mayoría de las tareas de NLP en los últimos años. \\
Estos modelan el contexto textual pero no de manera secuencial. Dada una palabra en una entrada de texto, se
prefiere mirar las demás palabras vecinas en el texto y representar cada palabra de acuerdo a el contexto
de las demás. Por ejemplo si el contexto habla de finanzas, entonces "banco" probablemente representaría una 
institución financiera. Por otro lado en el contexto de un rió, estaría relacionado con un montículo de tierra. 
Recientemente, los transformadores han sido utilizados para transferir aprendizaje con flujos de pequeñas tareas,
La transferencia de aprendizaje es una técnica en IA en la cual lo que se aprendió en resolver un problema es
aplicado para resolver problemas similares.

\begin{figure}[H]
    \centering
    \includegraphics[width=0.8\textwidth]{imagenes/Cap 2/tranformers.png}
    \caption{Diagrama de un Transformador}
    \label{fig:RNN}
    \cite{sowmya_practical_npl}
\end{figure}

\subsection{Autoencoders}
Los autoencoders son otro tipo de red neuronal utilizadas principalmente para aprender a representar la
entrada de forma comprimida en un vector. Por ejemplo, si queremos representar unas palabras en un vector, 
podemos aprender a mapear el texto en vectores y luego remapear para reconstruir la entrada original.
Esto es una forma de aprendizaje no supervisado ya que no se necesitan datos anotados para el entrenamiento.\\

\begin{figure}[H]
    \centering
    \includegraphics[width=0.8\textwidth]{imagenes/Cap 2/autoencoder.png}
    \caption{Representacion de un autoencoder}
    \label{fig:autoencoder}
    \cite{sowmya_practical_npl}
\end{figure}


%comparacion de metodos 

% TODO agregar aplicaciones 

\section{Chatbots}
% conceptos mas específicos
%---data adquisitcition


%----extraccion de texto

%----preprosesado


%----feature engineeniring 

%----modeling 


%---evaluation

%----post modelign 


\subsection{Concepto}
Los chatbots son programas que imitan la conversación humana utilizando la Inteligencia Artificial.\cite{UniversityRelatedFAQS}
\subsection{Características}
Existen diferentes tipos de chatbots, clasificados según su complejidad, objetivos o funciones, pero todos ellos cuentan con las siguientes cualidades según Nicol Radziwill y Morgan Benton \cite{evaluating_quality}
\begin{itemize}
    \item Rendimiento: se refiere a la eficiencia en la asignación de funciones y a la robustez que tienen en cuanto a la manipulación y a las entradas inesperadas.
    \item Funcionalidad: es capaz de interpretar, responder y ejecutar  correctamente las tareas demandadas.
    \item Humanidad: la conversación con el chatbot debe ser natural, lo mas parecida a la humana.
    \item Ética: genera confianza, respeta y protege la dignidad, y la privacidad de los usuarios.
    \item Accesibilidad: se encuentra disponible cuando el usuario quiera usarlo, además se refiere a que es capaz de detectar intenciones y significados
\end{itemize}
% documentación de rasa como referencia

\subsection{Aplicaciones}
Se pueden mencionar dos tipos de aplicaciones:
\begin{itemize}
    \item Asistentes Personales Virtuales: ofrecen servicios a los usuarios a través de texto o voz. Ejemplos: Siri(Apple), Google Assistant, Alexa (Amazon) y Cortana (Microsoft)
    \item Bots para el consumo específico: sus aplicaciones son muy variadas, puede utilizarse en el transporte, la salud, el clima, entretenimiento o incluso en la educación.
\end{itemize}
\subsection{Arquitectura} 
Para el diseño adecuado de cualquier sistema la mejor solución es dividirla en varias partes o subsistemas de acuerdo a un estándar.
En la figura \ref{fig:Arquitectura} se puede ver que existen dos bloques bien definidos, al primero se lo llama 'lado del cliente' que es la parte que interactúa con los usuarios; y el segundo bloque es el 'lado del servidor' encargado de procesar las peticiones del cliente.\\
\begin{figure}[h]
    \centering
    \includegraphics[width=\textwidth]{imagenes/Cap 2/Arquitectura.png}
    \caption{Arquitectura de los Chatbots}
    \label{fig:Arquitectura}
\end{figure}
\indent El  proceso inicia en la interfaz de usuario cuando el cliente realiza una solicitud, ésta es analizada en el bloque componente de comprensión de lenguaje, aquí se extrae toda la información necesaria y se deduce las intenciones del usuario.\\
\indent Una vez que el chatbot llegue a la mejor interpretación que puede, ejecuta las acciones solicitadas o recupera la información de su fuente de datos, que puede ser una base de datos propia o datos externos accedidos a través de APIs.\\
\indent Posteriormente se generan las respuestas lo más parecidas a las que daría una persona humana, para ello utiliza la información de intención y contexto proveída por el componente de análisis de mensajes del usuario.\\
\indent El componente de gestión de diálogo se encarga de solicitar información faltante, aclaraciones y hacer preguntas de seguimiento.\cite{Overview_of_chatbots}

\subsection{Plataformas de desarrollo}
\begin{itemize}
    \item DialogFlow:
        Es la plataforma de desarrollo de chatbots de Google, permite una fácil integración a aplicaciones móviles y web, también facilita bastante el diseño de la interfaz de usuario.\\
        Admite como entrada texto y voz, es capaz de responder a los clientes con texto o voz sintética.\\
        Existen dos versiones, Dialogflw CX utilizado para agentes grandes o muy avanzados y Dialogflow ES que es la versión estándar, ésta cuenta con una versión gratuita.\\
        El precio varia según la versión elegida y el tipo de entrada, Dialogflow ES cobra 0.002 USD por cada solicitud realizada por texto y 0.0065 USD si la entrada es un audio de hasta 15 segundos.\cite{Dialogflow}
    \item IBM Watson:
        IBM Watson  permite a los usuarios integrar sus chatbots en cualquier canal, sea web, aplicaciones o incluso una llamada.\\
        Es capaz de aprender los vocabularios de la industria,  términos coloquiales o dialectos regionales, admite entradas de voz y también de texto. \\
        Está diseñada para aprender sobre el camino, proporciona herramientas para detectar las tendencias y estas ayudan a asignar recursos de forma mas eficiente y eficaz.\\
        No es necesario escribir ni una sola linea de código ya que utiliza un entorno de 'arrastre y suelte' para construir los diálogos y una vez que lo adaptemos a nuestras necesidades, fácilmente lo incorporamos a la app con 'copiar y pegar'.\cite{IBMCloud2020}\\
        Cuenta con tres versiones: la versión mas básica, Lite, es gratuita pero muy limitada en sus funcionalidades.
        Luego vienen las versiones de paga, Plus y Enterprise con precios que van desde los 140 USD por mes.\cite{IBM_Price}
    \item Amazon Lex:
        Amazon Lex es la propuesta del gigante tecnológico Amazon, sirve para diseñar, crear, probar e implementar interfaces de conversación en las aplicaciones.\\
        Reconoce tanto entradas de texto como de voz, gestiona el contexto de las conversaciones de forma nativa y también permite una gran fidelidad en las interacciones de habla telefónica.\\
        La integración con plataformas se realiza de forma muy sencilla desde la consola de Amazon Lex , permite aplicaciones web, móviles y los servicios propios de Amazon como Amazon Kendra, Amazon Polly o AWS Lambda.\\
        Los precios varían según el tipo de servicio solicitado, El mas básico 'Interacción de respuesta y solicitud' cobra 0,004 USD por solicitud de voz y 0,00075 USD por solicitud de texto.\cite{Amazon_Lex}
    \item RASA Open Source:
    Es una plataforma de código abierto que proporciona procesamiento de lenguaje natural para convertir los mensajes de los usuarios en intenciones y entidades que los chatbots entienden, permite la gestión de los diálogos basándose en los mensajes de los usuarios y el contexto de la conversación.\\
    La integración a las aplicaciones mas comunes de mensajes y a los canales de voz se puede hacer de forma muy sencilla con los conectores ya incorporados, para conectar a las demás aplicaciones móviles o web se deben personalizar los conectores.\\
    RASA también tiene una versión de paga denominada RASA Enterprise, utilizada para el despliegue a escala, su precio varía de acuerdo a las necesidades de la organización y del proyecto.\cite{Rasa}
    \item Chatterbot:
    Chatterbot es una librería de Python que facilita la automatización de respuestas mediante  distintos algoritmos de aprendizaje automático, esto permite que una instancia de agente mejore su propio conocimiento de las posibles respuestas a medida que interactúa con humanos y otras fuentes de datos informativos\\
    Cada vez que un usuario introduce una frase, la librería guarda el texto que ha introducido y el texto al que responde la frase. A medida que ChatterBot recibe más entradas, aumenta el número de respuestas que puede dar y la precisión de cada respuesta en relación con la declaración introducida.\\
    Se puede integrar a las aplicaciones mediante APIs, además ChatterBot tiene soporte directo para la integración con el ORM de Django, esto facilita bastante para crear las páginas conversacionales.\cite{Chatterbot}\\
\end{itemize}

\indent Haciendo una comparación entre todas las herramientas analizadas, vemos que la creación de los chatbots con Dialogflow, IBM Watson y Amazon Lex es mucho mas sencilla, ya que tienen una madurez tecnológica muy alta, gran soporte y las interfaces son muy intuitivas, además no requieren de mucha programación. La mayor desventaja es que las versiones que cuentan con todas las funcionalidades son de paga, y las versiones gratuitas son muy limitadas y básicas, por lo que descartamos estas tres opciones.\\
\indent En cuanto a Chatterbot y RASA, la curva de aprendizaje puede ser un poco mayor porque no utilizan interfaces gráficas, todo es programado con Python, un lenguaje muy versátil que cuenta con numerosas librerias, esto permite tener mayor control sobre los chatbots porque se pueden manipular todos los ficheros y modificar las configuraciones, Si bien ambos están muy bien documentados, RASA destaca de Chatterbot por su comunidad, cuenta incluso con un foro donde participan los desarrolladores y usuarios dispuestos a brindar ayuda a todo aquel que las necesite.\\
\indent Teniendo en cuenta que no contamos con experiencia en el desarrollo de chatbots, se valora de sobremanera la comunidad, documentación y tutoriales con los que cuenta RASA, además que permite la integración con distintas plataformas y el lenguaje de programación que utiliza (Python) es de nuestro conocimiento, es por eso que concluimos con que RASA es una buena elección para llevar a cabo este proyecto.


    
\chapter[RASA]{RASA Open Source}
Rasa es un framework que permite la construcción de forma sencilla de chatbots personalizados, está
compuesta de dos librerias de código abierto, Rasa NLU y Rasa Core, denominadas en conjunto Rasa
Stack.
\indent
\section{Conceptos Básicos}
\begin{itemize}
	\item \textbf{intents (intenciones): }son las categorías, denominadas utterances creadas para
	      lo que el usuario está tratando de transmitir o lograr en una conversación, por ejemplo 'saludos'
	      donde se especifican las distintas formas de saludar. Las intenciones pueden ser divididas en
	      pequeñas subintenciones denominadas 'Retrieval Intent'.
	\item \textbf{entities (entidades): } las entidades son informaciones o palabras clave que
	      pueden ser extraídas de un mensaje para personalizar la conversación.
	\item \textbf{slots:} es un registro de datos que Rasa utiliza para guardar la información
	      proveída por el usuario en el curso de la conversación, un claro ejemplo del uso de este elemento
	      es almacenar el nombre del usuario para personalizar los mensajes.
	\item \textbf{responses (respuestas):} mensajes que los chatbots envían a los usuarios, estos
	      pueden ser dinámicos y con cualquier tipo de contenido como texto, imágenes, links, etc.
	\item \textbf{forms (formularios):} Un tipo de acción personalizada que pide al usuario varios
	      datos.
	\item \textbf{actions (acciones):} es un paso que toma el bot en la conversación por ejemplo,
	      llamar a una API o enviar una respuesta al usuario.\cite{Glossary}
\end{itemize}

\section{Cómo se lleva a cabo las conversaciones?}
Para llevar a cabo las conversaciones se utilizan las dos librerías del Rasa Stack.
\begin{itemize}
	\item \textbf{Rasa NLU: } En ella  se escriben los archivos de configuración, se elige el
	      pipeline y el modelo de entrenamiento para que deduzca las intenciones y posteriormente pueda
	      extraer las entidades disponibles.\\
	      \indent Puede ser basado en reglas o en redes neuronales, el primero suele ser mas ligero y no
	      necesita de muchos datos aunque no son buenos en tareas antes no vistas, mientras que el segundo
	      necesita de mas capacidad de cómputo y datos para entrenamiento, son mas flexibles que los basados
	      en reglas, ya que pueden aprender cosas que no han visto antes.
	\item \textbf{Rasa Core: } es el gestor de diálogos utilizado para crear modelos que sean
	      capaces de decidir que respuestas o acciones se ejecutarán de acuerdo a las entradas generadas por
	      el usuario.\\
	      \indent También puede ser basado en reglas, que es el enfoque mas tradicional, funciona muy
	      bien en muchos casos pero es difícil de expandir las conversaciones, también puede ser basado en
	      redes neuronales que escoge la siguiente acción basándose en la conversación y en los ejemplos del
	      entrenamiento.
\end{itemize}
\indent Básicamente, Rasa NLU se encarga de interpretar los mensajes y Rasa Core de de decidir que
acción tomar.\\
\indent Para asegurarnos de que una conversación funcione Rasa utiliza un proceso denominado
‘conversation-driven development’ que consiste en revisar manualmente las conversaciones para
detectar cualquier error cometido, agregar nuevos datos de entrenamiento, volver a entrenar el
modelo y probarlo nuevamente\cite{Introduction_to_Rasa}
\section[Instalación de RASA]{Instalación}
Primeramente crearemos un entorno virtual denominado 'venv' utilizando Python en una computadora
con Linux.

\begin{center}
	\framebox[10cm][c]{python3 -m venv ./venv}
\end{center}
Luego se activa el entorno virtual.
\begin{center}
	\framebox[10cm][c]{source ./venv/bin/activate}
\end{center}
Y por último se instala Rasa Open Source utilizando pip (requiere Python 3.7 o 3.8)
\begin{center}
	\framebox[10cm][c]{sudo pip3 install -U –user pip \&\& pip3 install rasa}
\end{center}

\section{Creación de un Proyecto}
\indent La creación de un nuevo proyecto se realiza con el comando 'rasa init', éste crea un
conjunto de carpetas y archivos así como se muestra en la figura \ref{fig:Estructura}.
\begin{figure}[h]
	\centering
	\includegraphics[width=\textwidth]{imagenes/cap3/4_Estructura del Proyecto.png}
	\caption{Estrucutra de los archivos}
	\label{fig:Estructura}
	\cite{Rasa}
\end{figure}
\subsection{Archivo nlu}
En él se encuentran datos estructurados que sirven para entrenar el modelo y luego extraer la
información de los mensajes del usuario. Estos datos son  las intenciones y entidades, también se
pueden agregar expresiones regulares y algunas tablas de búsqueda. \cite{NLU_Documentation}

\subsection{Archivo Rules}
En este archivo se definen las reglas, que no son mas que  tipo de datos de entrenamiento
encargados de describir partes de una conversación que siempre sigue el mismo
camino.\cite{Rules_Documentation}

\subsection{Archivo Stories}
Las historias son un tipo de datos de entrenamiento, se utiliza para entrenar modelos que puedan
generalizar las rutas de conversación. Las entradas del usuario son expresadas mediante intents, y
entitites si es necesario,  mientras que las respuestas del asistente son expresadas mediante
actions.\\
Los patrones que siguen las conversaciones podemos extraer de datos ya existentes o con la
herramienta de rasa 'interactive learning'.\cite{Stories_Documentation}

\subsection{Archivo config}
En el archivo config se definen el lenguaje y los componentes del pipeline, que forman parte de
Rasa NLU y las políticas a ser utilizadas, correspondiente a Rasa NLU.\\
El pipeline es el encargado de definir la dirección de flujo de datos entre los diferentes
componentes, Rasa nos permite configurar cada uno de ellos según nuestras necesidades, de tal forma
que podamos realizar las predicciones de las intenciones y la extracción de las entidades. Las
políticas forman parte de la gestión de diálogos, encargada de seleccionar la siguiente acción a
ser ejecutada.\cite{Configuration_Documentation}\\
La configuración del pipeline y las políticas son de suma importancia, por lo que se detallaran sus
componentes en la sección \ref{ch:Componentes}.

\subsection{Archivo credentials}
Aquí se definen los credenciales para las plataformas de voz y chat que el bot utiliza. Rasa cuenta
con algunos conectores preestablecidos para los canales mas conocidos como Facebook Messenger,
Telegram, Google Hangouts Chat o una pagina web propia.\cite{Credentials_Documentation}

\subsection{Archivo domain}
El archivo domain es un archivo de configuración donde se especifican las intenciones, entidades,
slots, respuestas, formularios y acciones que el bot debe saber.\cite{Domain_Documentation}

\subsection{Archivo endpoints}
Los endpoints son los enlaces a los servicios externos o internos que puede tener Rasa. En el se
definen los servidores que corren o en los que están alojadas las acciones personalizadas, al igual
que los modelos con los que se cuenta. También es aquí donde se especifican los tracker store,
utilizados para guardar las conversaciones, y los event broker, encargados de conectar el bot con
otros servicios que procesan los datos que llegan de las conversaciones.

\section{Componentes}\label{ch:Componentes}
% de https://rasa.com/blog/intents-entities-understanding-the-rasa-nlu-pipeline/
En esta sección estaremos describiendo el funcionamiento de los componentes utilizados en la
arquitectura de Rasa,
estos componentes son modulares y genéricos para lo que son sistemas de NLU modernos, pueden ser
propios del
entorno o proveídos por otras librerías de terceros para extender funcionalidades.

\begin{figure}[h]
	\centering
	\includegraphics[width=\textwidth]{imagenes/cap3/rasa_components.png}
	\caption{Componentes de un esquema NLU}
	\label{fig:Componentes-MLU}
	\cite{Rasa}
\end{figure}

\subsection{Tokenizadores}
Antes de poder ser procesada una porción de texto debe ser dividida en porciones mas pequeñas, para
esto se suele utilizar
un tokenizador (o tokenizer).	Este divide el texto en componentes de un vector.\\ Algunos
tokenizadores también
agregan información extra a los tokens que pueden ser usados para generar lemas, o sea extraer la
palabra que da el significado base a las palabras que pueden ser utilizados por el contador de
vectores.\\
Para el Inglés y el Español usualmente se usa WhiteSpaceTokenizer que separa en tokens cuando se
detectan espacios, para los idiomas que no precisen de espacios en blanco para separar las palabras
como el Coreano, Japones o Chino, se utilizan MitieTokenizer, en el caso del último tambien es muy
frecuente el uso de JiebaTokenizer.\cite{warmerdam_2022}

\begin{figure}[h]
	\centering
	\includegraphics[width=\textwidth]{imagenes/cap3/tokenization.png}
	\caption{Tokenización}
	\label{fig:tokenization-MLU}
	\cite{Rasa}
\end{figure}

\subsection{Caracterizadores}
Los caracterizadores generan pesos numéricos para ser consumidos por los modelos de ML. Existen
dos tipos principales de
características, las Dispersas(Sparse Features) que usualmente cuentan lo que pueden representar
subpalabras o características
léxicas y las Densas(Dense Features) estos suelen consistir en porciones preentrenadas, para que
estos se desempeñen correctamente se debe seleccionar un Tokenizador apropiado.
\cite{warmerdam_2022}
Todos los caracterizadores son presentados en la tabla \ref{tab:Caracterizadores}

\begin{table}[]
	\resizebox{\textwidth}{!}{%
		\begin{tabular}{|l|l|l|}
			\hline
			\textbf{Caracterizador}    & \textbf{Requisitos}     & \textbf{Tipo}
			\\ \hline
			MitieFeaturizer            & MitieNLP                & Dense featurizer
			\\ \hline
			SpacyFeaturizer            & Dense / Sparse Features & \begin{tabular}[c]{@{}l@{}}Logistic
				                                                       Regression de \\ scikit-learn\end{tabular} \\ \hline
			ConveRTFeaturizer          & Tokenization            & Dense featurizer
			\\ \hline
			LanguageModelFeaturizer    & Tokenization            & Dense featurizer
			\\ \hline
			CountVectorsFeaturizer     & Tokenization            & Sparse featurizer
			\\ \hline
			LexicalSyntactitFeaturizer & Tokenization            & Sparse featurizer
			\\ \hline
			RegexFeaturizer            & Tokenization            & Sparse featurizer
			\\ \hline
		\end{tabular}%
	}
	\caption{ Caracterizadores. Elaboración Propia}
	\label{tab:Caracterizadores}
\end{table}

\begin{figure}[h!]
	\centering
	\includegraphics[width=\textwidth]{imagenes/cap3/featurizers.png}
	\caption{Caracterización}
	\label{fig:feazturization-MLU}
	\cite{Rasa}
\end{figure}

\subsection{Clasificadores de Intención(Intent Classiffiers)}

Una vez que se generaron las características para todos los tokens y para toda la oración, podemos
pasarlos a un
modelo clasificador de intenciones. Rasa por defecto usa el modelo DIET que puede encargarse tanto
de
la clasificación de la intención y extracción de entidades. También puede aprender tanto de
características de Tokens como de oraciones. \cite{warmerdam_2022}
% Please add the following required packages to your document preamble:
% \usepackage{graphicx}
\begin{table}[]
	\resizebox{\textwidth}{!}{%
		\begin{tabular}{|l|l|l|}
			\hline
			\textbf{Clasificador}        & \textbf{Requisitos}     & \textbf{Utiliza}
			\\ \hline
			MitieIntentClassifier        & MitieNLP                & Clasificación multiclase con SVM
			\\ \hline
			LogisticRegressionClassifier & Dense / Sparse Features & Logistic Regression de scikit-learn
			\\ \hline
			SklearnIntentClassifier      & Dense Features          & \begin{tabular}[c]{@{}l@{}}SVM lineal
				                                                         optimizado con búsqueda \\ en cuadrícula\end{tabular} \\ \hline
			KeywordIntentClassifier      & Ninguno                 & Comparador de palabras clave
			\\ \hline
			DIETClassifier               & Dense Features          & Transformadores
			\\ \hline
			FallbackClassifier           & Intents                 & \begin{tabular}[c]{@{}l@{}}Requiere de un
				                                                         clasificador de \\ intenciones previo\end{tabular}    \\ \hline
		\end{tabular}%
	}

	\caption{Clasificadores de intenciones. Elaboracion propia}
	\label{IntentClassifier}
\end{table}
\begin{figure}[h]
	\centering
	\includegraphics[width=\textwidth]{imagenes/cap3/intent_classiffier.png}
	\caption{Clasificación de intenciones y entidades}
	\label{fig:intentclasification-MLU}
	\cite{Rasa}
\end{figure}

\subsection{Extracción de entidades}

Además de DIET, existen otros clasificadores basados en ML que pueden aprender como detectar
entidades, estos no son recomendados para todos los casos, también puede ser implementado un
extractor basado en Expresiones
Regulares(RegexEntityExtractor) \cite{warmerdam_2022}
% Please add the following required packages to your document preamble:
% \usepackage{graphicx}

\begin{table}[h!]
	\resizebox{\textwidth}{!}{%
		\begin{tabular}{|l|l|l|}
			\hline
			\textbf{Clasificador}   & \textbf{Requisitos} & \textbf{Utiliza}
			\\ \hline
			MitieEntityExtractor    & MitieNLP            & \begin{tabular}[c]{@{}l@{}}Clasificación multiclase
				                                                \\ con SVM\end{tabular} \\ \hline
			SpacyEntityExtractor    & SpacyNLP            & Modelo estadístico BILOU
			\\ \hline
			CRFEntityExtractor      & Tokens              & Campo aleatorio condicional (CRF)
			\\ \hline
			DucklingEntityExtractor & Ninguno             & Expresiones regulares
			\\ \hline
			DIETClassifier          & Dense Features      & Transformadores
			\\ \hline
			RegexEntityExtractor    & Ninguno             & Tablas de búsqueda
			\\ \hline
		\end{tabular}%
	}
	\caption{Extractores de entidades. Elaboración propia}
	\label{EntityExtractor}
\end{table}
\begin{figure}[h]
	\centering
	\includegraphics[width=\textwidth]{imagenes/cap3/regex_extractor.png}
	\caption{Extractor Regex}
	\label{fig:regex-extractor}
	\cite{Rasa}
\end{figure}

\subsection{Selectores}
Los selectores se encargan de predecir la respuesta de un conjunto de respuestas posibles para los
retrieval intents, son utilizados posteriormente por el gestor de diálogo para dar la respuesta mas
adecuada.
Utiliza la misma arquitectura y optimización que el DIETClassifier.

\subsection{Predicción de acciones}
\begin{figure}[h!]
	\centering
	\includegraphics[width=\textwidth]{imagenes/cap3/predicciones.png}
	\caption{Predicción de acciones.}
	\label{fig:regex-extractor}
	\cite{Rasa}
\end{figure}

Con el flujo NLU, se detectan las entidades e intenciones. Pero este flujo no predice la siguiente
acción en la
conversación. Para esto se utiliza el flujo de política. Las políticas aseguran el uso de
predicciones de NLU así como
también el estado presente de la conversaciones para decidir que acción tomar, pueden ser basado en
reglas o en aprendizaje automático.\\
\textbf{Políticas basadas en Aprendizaje Automático:}
\begin{itemize}
	\item \textbf{TED Policy:}
	      Es un conjunto de algoritmos desarrollados por RASA para la predicción de diálogo y
	      reconocimiento de entidades. Su arquitectura se basa en transformadores que convierten el diálogo
	      actual en un vector de diálogos, para compararlos con otros vectores en busca del mas cercano, a
	      partir de las acciones existentes.\cite{ConversationalAIwithRasa}
	\item \textbf{UnexpecTED Intent Policy:} Es una política auxiliar, tiene la misma arquitectura
	      que TEDPolicy pero éste aprende cuales son las intenciones mas probables a ser expresadas según el
	      contexto de la conversación. Siempre debe usarse en conjunto con al menos una otra
	      política.\cite{UnexpecTED}
	\item \textbf{Memoization Policy: }Esta política utiliza las historias y acciones de los datos
	      de entrenamiento y las guarda en un diccionario, si la conversación actual no coincide con ningun
	      ejemplo, predice un 0.0.\cite{MemoizationPolicy}
	\item \textbf{Augmented Memoization Policy: } Tiene las mismas funcionalidades de Memoization
	      Policy, pero además cuenta con un mecanismo que permite olvidar de forma aleatoria algunas partes
	      de la conversación, luego predice las acciones ya con la historia
	      reducida.\cite{AugmentedMemoizationPolicy}
\end{itemize}
\textbf{Políticas basadas en Reglas:}
\begin{itemize}
	\item \textbf{Rule Policy:} Realiza las predicciones basandose en reglas que se tienen en los
	      datos de entramiento.
\end{itemize}
Con cada interacción, las políticas definidas indican con un nivel de confianza cuál será la
siguiente acción a ser tomada, aquella que tiene obtiene el mayor resultado sera la que decida la
siguiente acción, en caso de que se prediga con la misma confianza, se tiene en cuenta la siguiente
asignación de importancia.
\begin{itemize}
	\item 6 - RulePolicy
	\item 3 - MemoizationPolicy o AugmentedMemoizationPolicy
	\item 2 -  UnexpecTEDIntentPolicy
	\item 1 - TEDPolicy
\end{itemize}

\section{Patrones de Conversación}
\subsection{Chitchat y FAQs}
Las preguntas frecuentes y los chitchats (conversaciones sobre temas no importantes) son casos
donde el asistente siempre tiene que responder de la misma forma, sin importar el contexto de la
conversación. El problema se encuentra en que si creamos intenciones y acciones para cada pregunta
que se realiza, las historias y reglas serán muy extensas, es por eso que estas preguntas
frecuentes las agrupamos en una 'retrieval intent' y se selecciona la respuesta correcta mediante
el componente 'Response Selector'.\\
Para poder utilizarlo se debe configurar apropiadamente el archivo config.yml. Este componente
utiliza la política basada en reglas 'RulePolicy', también caracterizadores y clasificadores de
intenciones por lo que debe ubicarse en el pipeline después de estos.
\subsection{Fallbacks}
Para los casos donde el usuario pregunta algo que está fuera del alcance del bot, establecemos una
intención llamada 'out of scope' a la que asociamos a una respuesta genérica y creamos una regla en
el archivo rules.yml.\\ Existen casos donde la confianza en la clasificación es muy baja, esto
implica que no se pueda predecir con buena confianza si se trata de una intención 'out of scope',
para ello Rasa tiene una opción llamada 'Fallback' que permite pedir al usuario que reformule su
pregunta para tratar nuevamente de predecir correctamente a que intención se refiere. \\
Para su utilización se debe agregar el 'FallbackClassifier' en el pipeline, crear respuestas por
defecto y actualizar las reglas.
\section{Pruebas}
Rasa cuenta con varias funciones para probar los diálogos, historias, el gestor de diálogos y el
procesamiento de mensajes.
\subsection{Validación de datos}
El comando 'rasa data validate' se encarga de verificar que no haya errores e inconsistencias en
los datos y configuraciones. Es recomendable ejecutar este comando antes de entrenar el modelo, ya
que si se encuentra algún problema, el entrenamiento también podría fallar.
\subsection{Evaluación del desempeño de la NLU}
Una practica usual al ejecutar aprendizaje automático es dividir aleatoriamente el conjunto de
datos en uno de entrenamiento y otro de pruebas. El bot utiliza el primer conjunto para aprender
las características necesarias para realizar las predicciones adecuadas, y el segundo conjunto para
evaluar el modelo mediante datos que aún no hayan sido vistos antes.\\
Rasa nos permite dividir los datos mediante el comando 'rasa data split nlu' que por defecto separa
los datos de entrenamiento/prueba en un 80/20, luego, para probar que tan bien entrenado se
encuentra el modelo utilizamos 'rasa test nlu' especificando cuales son los datos de entrenamiento
y prueba de la siguiente forma:\\
--nlu train\_test\_split/test\_data.yml

Otro método muy completo para evaluar el modelo es mediante validación cruzada o cross validation.
Este divide los datos en múltiples subconjuntos llamados 'folds' donde se van alternando los datos
de entrenamiento y pruebas

\section{Despliegue del sistema}
Según la documentación de Rasa, el mejor momento para desplegar una versión de prueba es tan rápido
cuando
se tenga un bot que cumpla los requisitos mínimos de los requisitos de diseño. De esta manera se
puede
tener pruebas de usuarios reales lo mas rápido posible y poder agregar estos casos.

\subsection{Arquitectura}

\begin{figure}[h]
	\centering
	\includegraphics[width=\textwidth]{imagenes/cap3/architecture_deploy.png}
	\caption{Arquitectura}
	\label{fig:deploy-architecture}
	\cite{Rasa}
\end{figure}

\subsubsection{Servicios}
El diagrama muestra tres categorías de servicios: Los morados son componentes de principales de
Rasa
Enterprise, los azules son los servicios principales de Rasa Open Source y los anaranjados son
servicios
de terceros.

Tanto los componentes de Rasa Open Source e Enterprise tienen bases de datos independientes. Los
eventos
datos de conversaciones fluyen desde los servicios de Rasa Open Source a Rasa Enterprise a través
del event broker.

\subsubsection{Servicios de Rasa Enterprise}

Los servicios de Rasa Enterprise, no son todos de libre uso, algunas partes se utilizan bajo un
modelo de
licencias. Pero estos si bien no son indispensables (pueden ser reemplazados por servicios libres
de terceros) hacen que el despliegue de el producto sea mucho mas sencillo y así como también su
desarrollo. Dicho esto pasamos a explicar los existentes en el diagrama, se tratan de los 3 mínimos
servicios para
ejecutar Rasa Enterprise estos deben ser todos de la misma versión para asegurar compatibilidad.

\begin{itemize}
	\item \textbf{Servicio de eventos(Event Service):} Consume datos del corredor de eventos(event

	      broker) y los archiva en la base de datos.
	\item \textbf{Servicio de migración de la Base de datos(DB migration service):}  Asegura que
	      los esquemas de la base de datos
	      esten actualizados con respecto a la versión actual de Rasa Enterprise.
	\item \textbf{Rasa-X:} Ejecuta las tareas del back-end y front-end. Archiva y recupera datos de

	      conversaciones, entrenamiento y meta-datos, como rótulos de conversaciones y banderas de
	      mensajes
	      en la base de datos de Rasa Enterprise. El front-end usa las entradas del back-end para brindar
	      una
	      interfaz amigable para el usuario.
\end{itemize}

\subsubsection{Servicios de Rasa Open Source}

Los servicios de Rasa Open Source se ejecutan de manera totalmente independiente de Rasa
Enterprise,
por otro lado Rasa Enterprise si depende de Rasa Open Source para manejar los datos de las
conversaciones,
entrenamiento y ejecución de modelos. En orden para que Rasa Enterprise pueda mostrar
conversaciones
tomadas por Rasa Open Source este publica los eventos de una conversación a el mismo event broker
al cual Rasa Enterprise esta consumiendo.

\subsubsection{Partes}

\begin{itemize}
	\item \textbf{rasa-production:} Es el servicio que ejecuta el modelo entrenado, usado para
	      parsear mensajes en intents y predecir acciones en conversaciones con el usuario, en un canal
	      o
	      en UI de Rasa Enterprise.
	\item \textbf{rasa-worker:} Es utilizado para servicios de segundo planos, como para entrenar
	      un
	      nuevo modelo.
	\item \textbf{app:} Es el servidor de acciones personalizadas, ejecutan acciones especificas
	      de la aplicación.
\end{itemize}

\chapter[IMPLEMENTACIÓN]{IMPLEMENTACIÓN}
En los capítulos anteriores presentamos implementaciones de alto nivel de la arquitectura de
sistemas conceptuales de chatbots, así como también arquitecturas recomendadas por la documentación
de Rasa. En el presente capítulo presentaremos los componentes elegidos y sus funciones en el
sistema.

\section{Docker}
Existen varias implementaciones de contenedores para servidores como Docker, containerd, OpenVZ y
HyperV. La más utilizada en la industria y con más variedad de servicios actualmente es
Docker por lo que decidimos elegir esta herramienta para el desarrollo y despliegue de servicios del
proyecto. Docker puede ser ejecutado sobre sistemas basados en Windows, Linux y macOS lo que
facilita la igualdad de condiciones entre ambientes de desarrollo y producción, ademas es de uso
libre y gratuito. \cite{alternativas_docker}

\subsection{Contenedor}

Los contenedores son la unidad más pequeña un sistema, es una entidad que se utiliza para aislar
cada componente del sistema base. Cada contenedor puede aislarse mediante funciones del sistema
operativo llamadas cgroups y cnames logrando así aplicaciones en entornos aislados (sandbox en
Ingles). \cite{Docker}

\begin{figure}[ht]
	\centering
	\includegraphics[width=\textwidth]{imagenes/cap4/docker-container.png}
	\caption{Contenedores}
	\cite{Docker}
	\label{fig:container_diagram}
\end{figure}

\subsection{Imagen de contenedor}

Una imagen de contenedor es el sistema de archivos aislados de todos los archivos necesarios para
ejecutar la aplicación, así como dependencias, configuraciones, ejecutables, etc. Así como también
variables de entorno y datos necesarios para ejecutar la aplicación. \cite{Docker}

\subsection{Redes}

Entre las ventajas de desplegar una aplicación por medio de contenedores Docker es que se pueden
comunicar entre ellos y también con servicios externos al entorno de Docker. Por defecto se pueden
crear varios tipos de configuraciones de red, pero por lo general se utilizan redes puente entre
los contenedores para que estos puedan comunicarse mutuamente entre ellos y solo exponiendo los
puertos necesarios para interactuar con el sistema en cuestión y esta es la opción utilizada en
nuestra implementación. \cite{Docker}

\subsection{Volúmenes}
Puesto que un contenedor no tiene un estado persistente sobre los datos que genera, se introducen
los volúmenes, son la forma recomendada de agregar persistencia de datos a un contendedor de
Docker.
\cite{Docker}

\begin{figure}[ht]
	\centering
	\includegraphics[width=\textwidth]{imagenes/cap4/docker-volume.png}
	\caption{Volumen}
	\cite{Docker}
	\label{fig:volume_diagram}
\end{figure}

\subsection{Construcción}
Las imágenes de Docker se construyen partir de instrucciones escritas en un archivo denominado
Dockerfile,
generalmente de parte de una imagen base de la cual sé la adiciona lo necesario para ejecutar la
aplicación.
\cite{Docker}

\subsection{Repositorios}
Un repositorio de imágenes Docker(Docker Registry) es un servidor que almacena y distribuye
imágenes versionadas generadas partir de un Dockerfile. Estos repositorios pueden ser públicos como
DockerHub que es el oficial de la comunidad de Docker como así también privado para un equipo de
desarrollo en una institución.
\cite{Docker}

\section{Componentes}

Cada componente del sistema se configuró en un contenedor de Docker con la excepción del servidor
NGINX que si estaba instalado sobre el sistema operativo del servidor.
\begin{figure}[ht]
	\centering
	\includegraphics[width=\textwidth]{imagenes/cap4/server.png}
	\caption{Componentes del sistema}
	\textbf{Fuente:} Elaboración propia.
	\label{fig:server_diagram}
\end{figure}

\subsection{Rasa Open Source}

El contendedor de Rasa Open Source ejecuta una imagen oficial proveída por Rasa disponible en los
repositorios de DockerHub\cite{DockerHub}, pero con el modelo entrenado y las configuraciones
particulares al proyecto. Es el único contenedor que tiene un puerto externo para servir al
usuario. También hace uso de los servicios de PosgresSQL y del servicio de acciones(action server).

\subsection{Servidor de acciones}

Servidor de acciones(acction server) es un servicio interno que ejecuta código escrito en Python
este utiliza una imagen oficial para servicios Python disponible en los repositorios de
DockerHub\cite{DockerHub}, estos son operaciones específicas para algunas acciones una respuesta no
estática, como por el ejemplo la acción relacionada con el cálculo de puntajes en el final
desacuerdo a los puntajes en los exámenes parciales.

\subsection{PostgresSQL}
En la documentación de Rasa, especifica que se necesita un almacenamiento para el registro de
eventos (event broker) para su posterior análisis o también para el procesamiento de los eventos
por otros
servicios. Dependiendo de la demanda y el flujo de datos se pueden elegir Pika, Kafka y asi como
el uso de Bases de datos basadas en SQL para el almacenamiento. \cite{event_broker} Como
contábamos con experiencia previa en el uso de bases de datos utilizamos PostgresSQL que era una
implementación ya conocida, ademas para el desarrollo las funcionalidades adicionales de  si bien
podrían haber sido útiles agregarían complejidad innecesaria al sistema en fases iniciales y pueden
ser adicionados posteriormente.
PostgresSQL es un motor de base de datos del tipo relacional\cite{postgresql} el cual se configuró
a partir de una imagen oficial de PostgresSQL disponible en los repositorios de
DockerHub\cite{DockerHub}. Aparte de sus funciones de base de datos de sistema, también ejecutar un
trabajo periódico(cada una hora) para realizar una copia actualizada de los contenidos de la tabla
Eventos a un archivo separado por comas(csv) que se utiliza para retroalimentar las conversaciones
y generar más datos para mejorar el modelo.

\subsection{NGINX}
Para poder servir los archivos y la redirección interna de  las peticiones al servidor de Rasa Open
Source se necesita un servidor WEB con estas características. Existen  varias implementaciones que
pueden ser útiles como Apache, Caddy y NGINX. El más utilizado en la industria es NGINX por lo que
elegimos por su facilidad para configuración, amplia disponibilidad de documentación y asi como
también ser el más eficiente entre las demás opciones. \cite{web_servers}
NGINX, es un servidor web que también puede ser usado como proxy reverso, que implica redirigir el
tráfico a puertos internos y también para servir archivos que fueron las funciones utilizadas para
el proyecto. Así como también puede ser utilizado como balanceador de carga, mail proxy y HTTP
cache, entre otras funciones \cite{NGINX}
El servidor NGINX redirige el tráfico a la instancia de Rasa Open Source y si como también sirve el
archivo de
la copia más reciente de la tabla Eventos para agilizar las verificaciones de las respuestas y
preguntas recibidas.

\section{Recursos}
Para la puesta en producción se necesita un servidor que este disponible con acceso a internet y a
una IP publica, dada la amplia variedad de opciones de proveedores de nubes publicas, las ventajas
en disponibilidad, accesibilidad, facilidad cambio de prestaciones y el bajo
coste de mantenimiento que tienen actualmente en comparación de implementar un servidor propio
local optamos por usar un servicio en la nube. \cite{cloud_providers}

Se eligió el proveedor DigitalOcean, se necesito una instancia de Droplet alojado en Nueva
York del proveedor DigitalOcean con un procesador con un 1vcpu(procesador virtualizado), 1 GB de
RAM, 25 GB de almacenamiento y con un costo de entre 4 y 6 dólares americanos al mes dependiendo
del tráfico presentado. Por las limitaciones de memoria del servidor se configuró también 4 GB de
espacio de intercambio(SWAP) en el espacio de almacenamiento.

\section{Telegram}
Entre las opciones de interfaz existen tres opciones populares, implementar un cliente web o
aplicación Movil, o integrarlo dentro de plataformas de mensajería existentes. Para evitar agregar
complejidad al sistema decidimos optar por el uso de una plataforma existente. Si bien la plataforma
de mensajería más utilizada en nuestro medio es WhatsApp, esta presenta
costos para el uso de su interfaz de programación por lo que fue descartada para la implementación del
proyecto, la segunda más utilizada es Telegram que debido a su una amplia comunidad
para desarrollo de soluciones dentro de la plataforma puesto que es bastante sencillo
usar su servicio para conectar a implementaciones de chatbots, ademas no requiere costos para el
uso de su interfaz de programación por lo que consideramos que era la mejor opción para
publicar la solución a los usuarios finales.
\cite{botfather}
\chapter[ANÁLISIS]{ANÁLISIS DE RESULTADOS}
En este capítulo se presentan los resultados obtenidos en la implementación del chatbot utilizando
la plataforma Rasa OpenSource para responder preguntas frecuentes de estudiantes de la Facultad de
Ingeniería.  Además, se analizan los datos recopilados durante el entrenamiento del modelo, se
discuten las limitaciones, evaluara la efectividad del chatbot, asi como tambien posibles mejoras
en la implementación del chatbot.

\section{Configuraciones ulilizadas}
Dado el interés en que los resultados obtenidos sean replicables, en primer lugar, se explicarán
los componentes seleccionados para el entrenamiento y despliegue del modelo.
\begin{itemize}
	\item \textbf{WhitespaceTokenizer}: componente que divide el texto en palabras
	      individuales, utilizando espacios en blanco como delimitador.
	      \cite{Configuation_Documentation}
	\item \textbf{RegexFeaturizer}: componente que crea características basadas en expresiones
	      regulares. Esto puede ser útil para detectar patrones en el texto.
	      \cite{Configuration_Documentation}
	\item \textbf{LexicalSyntacticFeaturizer}: componente que combina características léxicas y
	      sintácticas para crear una mejor representación del texto. Utiliza etiquetas de
	      partes del discurso
	      y etiquetas de análisis de dependencia para crear características.
	      \cite{Configuration_Documentation}
	\item \textbf{CountVectorsFeaturizer}: componente que crea una representación dispersa de
	      bolsa de
	      palabras del texto. Puede ser utilizado para crear características para la
	      clasificación de
	      intenciones o la extracción de entidades. \cite{Configuration_Documentation}
	\item \textbf{DIETClassifier}: componente que combina una red neuronal recurrente con un
	      transformador para realizar la clasificación de intenciones y el reconocimiento de
	      entidades.
	      Utiliza múltiples fuentes de información, como incrustaciones de palabras,
	      incrustaciones de
	      caracteres y etiquetas de partes del discurso. \cite{Configuration_Documentation}
	\item \textbf{EntitySynonymMapper}: componente que mapea las entidades a su forma canónica.
	      Esto
	      puede ser útil para manejar variaciones en la forma en que se expresan las entidades
	      en el texto. \cite{Configuration_Documentation}
	\item \textbf{ResponseSelector}: componente que selecciona una respuesta basada en la
	      entrada del
	      usuario. Utiliza un enfoque basado en recuperación, donde coincide la entrada del
	      usuario con un
	      conjunto de respuestas predefinidas. Puede ser útil para manejar preguntas frecuentes
	      o
	      conversaciones informales.\cite{Configuration_Documentation}
	\item \textbf{FallbackClassifier}: componente que clasifica los mensajes como fallback si
	      no
	      coinciden con ninguna de las intenciones en el modelo. Puede ser útil para manejar
	      mensajes fuera
	      de contexto o solicitudes que el modelo no está entrenado para
	      manejar.\cite{Configuration_Documentation}
\end{itemize}
\section{Conjunto de Datos propio}

Inicialmente, se creó un conjunto de datos con las posibles preguntas frecuentes de los alumnos,
una vez que el Bot estuvo en funcionamiento e integrado a Telegram se recolectaron las preguntas
que realmente tienen los estudiantes de la FIUNA, estas fueron guardadas en una base de datos como
eventos, teniendo mucha información innecesaria para el conjunto de datos. Para un mejor manejo y
limpieza de los datos se utilizó Python con ayuda de la librería Pandas. Posteriormente se guardó
el dataframe creado en un archivo de Excel donde se analizaron y clasificaron un total de 1051
entradas en 141 intenciones distintas, De entre ellas, se identificaron las 15 intenciones más
recurrentes.

\begin{figure}[H]
	\centering
	\includegraphics[width=\textwidth]{imagenes/cap5/Top15intents.jpeg}
	\caption{Top 15 Intenciones}
	\label{fig:Top15intents}
\end{figure}

\section{Validación de los datos e Historias}
Rasa cuenta con varias funciones para probar los diálogos, historias, gestor de diálogos y el
procesamiento de mensajes, de tal forma a encontrar errores o inconsistencias antes de realizar el
entrenamiento.

\subsection{Validación de datos}
El siguiente comando se encarga de verificar que no haya errores e inconsistencias en los datos y
configuraciones.

\begin{center}
	\framebox[10cm][c]{rasa data validate}
\end{center}

Es recomendable ejecutarlo antes de entrenar el modelo, ya que si se encuentra algún problema, el
entrenamiento también podría fallar.

\subsection{Evaluación del desempeño de la NLU}
Una practica usual al ejecutar aprendizaje automático es dividir aleatoriamente el conjunto de
datos en uno de entrenamiento y otro de pruebas. El bot utiliza el primer conjunto para aprender
las características necesarias para realizar las predicciones adecuadas, y el segundo conjunto para
evaluar el modelo mediante datos que aún no hayan sido vistos antes.\\
Rasa nos permite dividir los datos mediante el comando:

\begin{center}
	\framebox[10cm][c]{rasa data split nlu}
\end{center}

Por defecto, Rasa separa los datos de entrenamiento/prueba en un 80/20, luego, para probar que tan
bien entrenado se encuentra el modelo utilizamos 'rasa test nlu' especificando cuales son los datos
de entrenamiento y prueba de la siguiente forma:\\

\begin{center}
	\framebox[10cm][c]{    --nlu train\_test\_split/test\_data.yml}
\end{center}

Rasa test proporciona herramientas que facilitan la detección y corrección de errores, incluye una
matriz de confusión, un archivo .json de reporte, un histograma de confianza y un archivo .json de
errores en caso de que existan.\\
La matriz de confusión es una herramienta fundamental que permite evaluar el rendimiento de un
modelo, permite identificar los falsos positivos y falsos negativos, nos muestra en su eje vertical
las etiquetas reales y en el eje horizontal las etiquetas predecidas, permitiendo identificar si
existen errores de clasificación.\\
Además, el script de rasa test guarda estos errores de clasificación en un archivo .json lo que
facilita el depurado y corrección de errores para mejorar la calidad de la clasificación.\\
El Histograma nos permite visualizar las predicciones del modelo y la confianza que ha sido
otorgada a cada intención o entidad.\\
Las predicciones correctas se encuentran en la parte izquierda del gráfico y son representadas en
color azul, mientras que las incorrectas se encuentran a la derecha y son de color rojo.\\
La ubicación de cada predicción en el eje horizontal del histograma representa el número de
muestras, y en el eje vertical la confianza con la que el modelo ha realizado su
predicción. \cite{interpretacion_graficos}

\subsection{Clasificador de Intenciones}

\begin{figure}[H]
	\centering
	\includegraphics[width=\textwidth]{imagenes/cap5/intent_confusion_matrix.png}
	\caption{Matriz de Confusión de Intenciones}
	\label{fig:intent_matriz}
\end{figure}

\begin{figure}[H]
	\centering
	\includegraphics[width=\textwidth]{imagenes/cap5/intent_histogram.png}
	\caption{Histograma de confianza en la clasificación de Intenciones}
	\label{fig:intent_histograma}
\end{figure}

Al analizar la matriz de confusión \ref{fig:intent_matriz} y el histograma
\ref{fig:intent_histograma} podemos verificar que todas las intenciones fueron clasificadas
correctamente con una confianza superior a 0.98, indicando que el modelo es bastante efectivo en su
tarea de clasificación.
\subsection{Extracción de entidades}
Al ver los gráficos del extractor, tanto en la matriz de confusión de entidades
\ref{fig:entity_matriz} como en el histograma \ref{fig:entity_histograma} se encuentra que el
modelo se confunde con dos entidades, revisando el reporte de errores se encuentra que los
extractores DIETClassifier y EntitySynonymMapper reconocen las entidades por separado, esto genera
el error pero no afecta al rendimiento del modelo o a la correcta selección de una respuesta.

\begin{figure}[H]
	\centering
	\includegraphics[width=\textwidth]{imagenes/cap5/DIETClassifier_confusion_matrix.png}
	\caption{Matriz de Confusión del extractor de entidades}
	\label{fig:entity_matriz}
\end{figure}

\begin{figure}[H]
	\centering
	\includegraphics[width=\textwidth]{imagenes/cap5/DIETClassifier_histogram.png}
	\caption{Histograma de confianza del extractor de entidades}
	\label{fig:entity_histograma}
\end{figure}

\subsection{Selección de Respuestas}
Podemos observar en el histograma \ref{fig:response_histograma} de la selección de respuestas que
no se predijo erróneamente ninguna respuesta, la matriz de confusión en este caso no es de mucha
utilidad ya que son bastantes respuestas y complica la visibilidad de las etiquetas, en caso de que
existiese un error podremos encontrarlo en el reporte .json.

\begin{figure}[H]
	\centering
	\includegraphics[width=\textwidth]{imagenes/cap5/response_selection_histogram.png}
	\caption{Histograma de confianza del seleccionador de respuestas}
	\label{fig:response_histograma}
\end{figure}

\section{Posibles Mejoras}
\begin{itemize}
	\item Integrar a sistemas existentes de la Facultad de Ingeniería.
	\item Autenticar usuarios a los sistemas de la Facultad de Ingeniería para obtener datos
	      específicos.
	\item Implementación de interfaz propia o utilización de elementos interactivos como
	      botones
	      para respuestas rápidas.
	\item Se puede agregar respuestas de la chatbots públicos como	ChatGPT \cite{api_chatgpt}
	      para preguntas
	      fuera del
	      contexto de la Facultad de Ingeniería.
\end{itemize}

\chapter[CONCLUSIONES]{CONCLUSIONES}

Primeramente se hizo una revisión de las tecnologías y plataformas existentes para el desarrollo de chatbots,
mediante una comparativa finalmente se optó por Rasa Open Source por ser de código libre, 
gran capacidad de personalización y además poseer buena documentación.
Para el entrenamiento inicial del modelo se utilizó un conjunto de datos propio, una vez que el modelo pudo ser desplegado en un servidor público
y el Bot haya estado disponible para los alumnos mediante una API de telegram, se recopilaron las preguntas frecuentes que tienen los estudiantes de la facultad de Ingeniería.
Las configuraciones del bot que contienen algoritmos de IA fueron seleccionadas de acuerdo a las necesidades del proyecto, estos fueron
entrenados y probados utilizando el conjunto de datos propio junto con el creado a partir de las preguntas de los alumnos.

%%%bibliografía%%%%%%%%%
%Cargar los libros en formato bibtex en el archivo fiuna.bib
\formatoIndice
\bibliographystyle{fiuna}
%\nocite{*}
\bibliography{fiuna}   

%%%anexos%%%%%%
%En caso de no tenerlos, comentar la siguiente línea
\appendixpageoff
\begin{appendices}
%Agregar los apéndices

\chapter*{Anexo A: Comandos de Rasa}
\addcontentsline{loa}{appendix}{Anexo A: Comandos de Rasa}
% Please add the following required packages to your document preamble:
% \usepackage{multirow}
% \usepackage{graphicx}
\begin{table}[h]
\resizebox{\textwidth}{!}{%
\begin{tabular}{|l|l|l}
\cline{1-2}
\multicolumn{1}{|c|}{\textbf{Comando}} & \multicolumn{1}{c|}{\textbf{Efecto}}                                                                                                    &  \\ \cline{1-2}
\multirow{2}{*}{rasa init}             & \multirow{2}{*}{Crea un nuevo proyecto con un ejemplo de entrenamiento, acciones y archivos de configuracion.}                          &  \\
                                       &                                                                                                                                         &  \\ \cline{1-2}
\multirow{2}{*}{rasa train}            & \multirow{2}{*}{Entrena un modelo utilizando datos NLU y lo guarda en ./models.}                                                        &  \\
                                       &                                                                                                                                         &  \\ \cline{1-2}
\multirow{2}{*}{rasa interactive}      & \multirow{2}{*}{Empieza una sesion interactiva de aprendizaje para crear un dataset nuevo de entrenamiento, chateando con el asistente} &  \\
                                       &                                                                                                                                         &  \\ \cline{1-2}
\multirow{2}{*}{rasa shell}            & \multirow{2}{*}{Carga el modelo entrenado y permite conversar con tu asistente en la linea de comando}                                   &  \\
                                       &                                                                                                                                         &  \\ \cline{1-2}
\multirow{2}{*}{rasa run}              & \multirow{2}{*}{Inicia un servidor con el modelo entrenado}                                                                             &  \\
                                       &                                                                                                                                         &  \\ \cline{1-2}
\multirow{2}{*}{rasa run actions}      & \multirow{2}{*}{Inicia un servidor de acciones utiliando el SDK de Rasa}                                                                &  \\
                                       &                                                                                                                                         &  \\ \cline{1-2}
\multirow{2}{*}{rasa visualize}        & \multirow{2}{*}{Genera una representación visual de las historias}                                                                      &  \\
                                       &                                                                                                                                         &  \\ \cline{1-2}
\multirow{2}{*}{rasas test}            & \multirow{2}{*}{Prueba un modelo entrenado de Rasa en cualquier archivo que empiea con test\_.}                                         &  \\
                                       &                                                                                                                                         &  \\ \cline{1-2}
\multirow{2}{*}{rasas data split nlu}  & \multirow{2}{*}{Realiza una separacion 80/20 de los datos de entrernamiento}                                                            &  \\
                                       &                                                                                                                                         &  \\ \cline{1-2}
\multirow{2}{*}{rasa data convert}     & \multirow{2}{*}{Convierte los datos de entrenamiento entre diferentes formatos}                                                         &  \\
                                       &                                                                                                                                         &  \\ \cline{1-2}
\multirow{2}{*}{rasa data migrate}     & \multirow{2}{*}{Migra del dominio 2.0 al formato 3.0}                                                                                   &  \\
                                       &                                                                                                                                         &  \\ \cline{1-2}
\multirow{2}{*}{rasa data validate}    & \multirow{2}{*}{Controla que los datos del  dominio, de la NLU y de la conversación no sean incoherentes}                               &  \\
                                       &                                                                                                                                         &  \\ \cline{1-2}
\multirow{2}{*}{rasa export}           & \multirow{2}{*}{Exporta conversaciones de un almacén de seguimiento a un corredor de eventos.}                                          &  \\
                                       &                                                                                                                                         &  \\ \cline{1-2}
\multirow{2}{*}{rasa valuates markers} & \multirow{2}{*}{Extrae los marcadores de un almacén de seguimiento existente.}                                                          &  \\
                                       &                                                                                                                                         &  \\ \cline{1-2}
\multirow{2}{*}{rasa x}                & \multirow{2}{*}{Inicia Rasa X en modo local}                                                                                            &  \\
                                       &                                                                                                                                         &  \\ \cline{1-2}
\multirow{2}{*}{rasa -h}               & \multirow{2}{*}{Muestra los comandos disponibles}                                                                                       &  \\
                                       &                                                                                                                                         &  \\ \cline{1-2}
\end{tabular}%
}
\end{table}


\chapter*{Anexo B: El uso de plantillas}
\addcontentsline{loa}{appendix}{Anexo B: El uso de plantillas}

Aquí el contenido del anexo B



\end{appendices}
\end{document}
